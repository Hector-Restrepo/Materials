%%%%%%%%%%%%%%%%%%%%%% Stock Preamble %%%%%%%%%%%%%%%%%%%%%%%%%%%%%%%%%%

\input{../LaTeX/undergrad_preamble.tex}

%%%%%%%%%%%%%%%%%%%%%%Headers and Footers%%%%%%%%%%%%%%%%%%%%%%%%%%%%%%%%%%
\pagestyle{headandfoot}
\runningheadrule
\firstpageheadrule
\firstpageheader{\includegraphics[width=0.25\textwidth]{../Figures/stern_black1.pdf}}{}{Data Bootcamp: Getting Started}
\runningheader{}{}{}
\runningfooter{}{}{}

\begin{document}
\bigskip
\centerline{\Large \bf Topic Outline:  Data + Python}
\medskip
\centerline{Revised: \today}


\section*{Materials}

\begin{itemize}
\item  Today's handouts:  Syllabus, this outline, red/green stickers
\item  All posted on website (except the stickers).
\end{itemize}

\section*{About the course}

\begin{itemize}
\item Data + Python = Magic!
\begin{itemize}
\item Arthur C. Clarke, Jessica, Tim
\end{itemize}

\item What?
\begin{itemize}
\item ... are you doing here?
\item Skills are nice, coding is literacy for the modern age
\item Something to show potential employers
\end{itemize}

\item Why?
\begin{itemize}
\item Why data?
\item Why code?
\item Why Python?
% general purpose language, great data tools, open source and free, community
\item Why bootcamp?
\item Why me?
\item Why you?
\end{itemize}

\item Things we believe
\begin{itemize}
\item Anyone can do this.  Target audience is {programming newbies --- with courage}.
\item It's ok to be lost.  We've all been there, it's not permanent.
%\item We're here to help.  But you need to tell us when you're lost.
\item This is fun.  Really.
%It's an amazing feeling to be able to do cool things in minutes.
\end{itemize}


\item Rules to live by
\begin{itemize}
\item Don't panic.  It will seem overwhelming at first, but stick with it and you'll be fine.
\item One step at a time.  Don't rush this.  In six weeks you'll know a lot.
\item Learn by doing.  Same directions as Carnegie Hall, no shortcuts.
%\item Let your nerd flag fly.  Learn to love xkcd.
\item Ask for help.  Don't be a hero, let us know if you could use some help.
\end{itemize}

\needspace{2\baselineskip}
\item Course materials
\begin{itemize}
\item Required:  practice, exam, project
\item Google ``nyu data bootcamp''
\item Website, topic list \& links  (thanks, Spencer):\\ \url{https://nyu.data-bootcamp.com/undergrad_outline/}  (bookmark me!)
\item Book: \\ \url{https://www.gitbook.com/book/nyudatabootcamp/data-bootcamp/details}
\item Data page: \\
\url{https://nyu.data-bootcamp.com/data/}
\item NYU Data Bootcamp GitHub repository:  \url{https://github.com/NYUDataBootcamp/Materials}
\item My Class Specific GitHub repository: \url{https://github.com/mwaugh0328/data_bootcamp_spring_2018}
\end{itemize}

\item You
\begin{itemize}
\item Come to class
\item After class:  {\bf write} and {\bf read}
\item Practice
\item Have fun
\end{itemize}
\end{itemize}

\section*{Setting up your computer}
\begin{itemize}
\item Create \verb|Data_Bootcamp| directory/folder on your computer. This is a place for you to save stuff to, work from, etc.
\item So do this\ldots
\begin{itemize}
\item Put red sticker on your laptop
\item Find your main hard drive, on PCs typically the ``C'' drive
\item In the C drive, create a folder called \verb|Data_Bootcamp|
\item Replace red sticker with green!
\end{itemize}
\item Now if you every have to call this file you know it is \verb|c:\Data_Bootcamp|
\end{itemize}

\section*{\href{https://www.anaconda.com/download/}{Anaconda}}

\begin{itemize}
\item Install the Anaconda distribution
\begin{itemize}
\item Put red sticker on your laptop
\item Distribution?
\item Google ``anaconda download'' or borrow a USB drive
\item Download or copy installer to your computer --- {\bf Python 3.6!}
\item Run installer
\item ONLY FOR PC USERS: on Windows Advanced Installation options, click the box ``Add Anaconda to my PATH environment"
\item Replace red sticker with green when installation is complete
\end{itemize}

\item Environments
\begin{itemize}
\item Environments?  (Analogy:  Word is an environment for creating Word docs.)
\item Jupyter:  environment for creating IPython notebooks, which combine code with text and output
\end{itemize}
\end{itemize}




\section*{Run test program -- twice}

\begin{itemize}
\item Test program code:

\vspace{-0.1in}
\begin{verbatim}
"""
Test program for Data Bootcamp course @ NYU Stern
"""
import sys

print('Welcome to Data Bootcamp!')
print('Python version:')
print(sys.version)
\end{verbatim}

\needspace{2\baselineskip}
\item Run test program in Jupyter Notebook
\begin{itemize}
\item Put red sticker on your laptop
\item From Terminal/Command prompt type \verb|jupyter notebook|
\item Look around...
\item Enter test program in a code cell
\item Click on \verb|Untitled| and rename the file as \verb|bootcamp_test|
\item Click on save button
\item Click on code cell, press \verb|shift + enter|
\item Look for correct output (last line should be {\tt 3.6.x etc})
\item Switch to green sticker if it works
\end{itemize}

% Not going to talk about Jupyter now...
%\item Run test program in Jupyter
%\begin{itemize}
%\item Put red sticker on your laptop
%\item From Launcher, launch Jupyter (labelled ``ipython-notebook'')
%\item Navigate to \verb|Data_Bootcamp| directory
%\item Open a new IPython notebook (New, Python 3)
%\item Change name from {\tt Untitled} to \verb|bootcamp_test|
%\item Look around (toolbar, menubar, code cells)
%\item Enter test program in code cell
%\item Run program (Cell, Run All)
%\item Look for correct output (last line should be {\tt 3.5.x etc})
%\item Switch to green sticker if it works
%\end{itemize}

\item Jupyter Notebook startup summary
\begin{itemize}
\item Open Terminal/Command and type \verb|jupyter notebook|
\end{itemize}
\end{itemize}


\section*{\href{https://github.com}{GitHub}}
\begin{itemize}
\item What I will use it for\ldots
\begin{itemize}
\item Source of ALL course materials
\item Place for you to grab materials on the fly. Save files by cut and paste, clever save as, or "Raw" (ask about this)
\end{itemize}
\item You need to create an account and email me your username
\item What you will use it for\ldots
\begin{itemize}
\item Think of this like an artists portfolio. Here you can post your code and projects and show potential employers, family, friends what you have done.
\end{itemize}
\item Now lets use it
\begin{itemize}
\item Put red sticker on your laptop
\item Sign into github (if you do not have an account, create one)
\item Create a new repository and name it \verb|my_first_repository|,
\item place the \verb|bootcamp_test.ipnyb| file you created in it,
\item Great job! Replace red sticker with green!\\
\end{itemize}
\end{itemize}


\section*{Practice and review}


Put red sticker on your laptop, replace with green when you're done.
Discuss with your neighbor.
Raise your hand if you could use some help.

\begin{enumerate}

\item Fill in the blanks in this table:
%relating ``environments'' to the files they are related to:

\begin{center}
\begin{tabular}{cc}
\toprule
Environment & File or Object \\
\midrule
MS Word  & Word document  \\
 & Excel file     \\
iTunes & \\
Typewriter & \\
Jupyter & \\
 & .py \\
\bottomrule
\end{tabular}
\end{center}

\item Run the \verb|maddison_data_input.ipnyb| Python code example.
\begin{itemize}
\item Go to my \verb|data_bootcamp_fall_2018| GitHub repository (link above).
\item Click on week1, get \verb|maddison_data_input.ipnyb|
\begin{itemize}
\item Right click on raw
\item Click \verb|save link as| then your folder window with directory should pop open, save it in your data
\end{itemize}
\item Open file in Jupyter.
\item Run it by clicking on \verb|Cells|, then select \verb|All|
\item What do you see?
\end{itemize}

\end{enumerate}

\section*{After class}

\begin{itemize}
\item Required
\begin{itemize}
\item Read Syllabus and Project Guide.
\item Mark Due Dates on your calendar.
\item Skim chapters 1-3 of the book.
\end{itemize}
\item Recommended
\begin{itemize}
\item If you haven't already:  join the discussion group.
\item Explore the website.  Make sure you can find the book, due dates,
topic outlines, assignments, and data sources.
\item Post a link to an interesting graph on the discussion group.
\item Look through the IPython notebook \verb|bootcamp_examples.ipynb|
in the {\tt Code/IPython} directory of the GitHub repo.
What graphs interest you?  What data?
Do they suggest anything else you might explore?
\end{itemize}
\end{itemize}

\end{document}

