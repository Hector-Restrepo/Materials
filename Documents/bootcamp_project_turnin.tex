%%%%%%%%%%%%%%%%%%%%%% Stock Preamble %%%%%%%%%%%%%%%%%%%%%%%%%%%%%%%%%%

\input{../LaTeX/undergrad_preamble.tex}

%%%%%%%%%%%%%%%%%%%%%%Headers and Footers%%%%%%%%%%%%%%%%%%%%%%%%%%%%%%%%%%
\pagestyle{headandfoot}
\runningheadrule
\firstpageheadrule
\firstpageheader{\includegraphics[width=0.25\textwidth]{../Figures/stern_black1.pdf}}{}{Data Bootcamp: Delivering Your Project}
\runningheader{}{Data Bootcamp: Delivering Your Project}{}
\runningfooter{}{}{}

\begin{document}

\bigskip\bigskip
\centerline{\Large \bf Data Bootcamp:  Delivering Your Project}
\centerline{Revised: \today}

\section*{Overview}

One of our goals is for you to produce a piece of work you can use to demonstrate your data and programming skills to potential employers. The end product should be posted on your GitHub account in a repository so that
you can show others what you've done. You will also be able to see what your classmates have done.

Below are the step by step details and format that I ask of you to post your project in. Not following these instructions exactly will result in a 10 percent reduction in your project score.

\textbf{Due Date: Midnight, at the end of the day December 21st}

\textbf{If you are working in a group, each individual must post the group project on their own GitHub site.} So if Jim and Mary are working together, then Jim posts the project on his site, Mary post the project on her site.

\section*{Step by Step Instructions}

\begin{enumerate}

\item Create a repository on your GitHub site. This repository must be named:

 {\tt Data\_Bootcamp\_Final\_Project}

 \bigskip

\item Populate the repository on your GitHub site with the following items:
\begin{itemize}

\item A readme file (GitHub should populate a blank one for you), that has \textbf{very specific language} specified on the next page. I will not grade your project until this step is complete.

\bigskip

\item Your Jupyter Notebook. This should be titled in the following way:

 {\tt Lastname\_Final\_Project.ipynb}

\bigskip

\item A .pdf file of your Jupyter Notebook. This should be titled in the following way:

 {\tt Lastname\_Final\_Project.pdf}

How to create this: Click on File/Print Preview/Print/ and then there you should be able to select ``Save as PDF''

\end{itemize}

\bigskip

\item Email a a link to your repository to the NYU databootcamp email \href{mailto:nyu.databootcamp@gmail.com}{nyu.databootcamp@gmail.com}

\end{enumerate}

All of this must be done prior to midnight, at the end of the day December 21st. Finally, \textbf{do not modify} your work after the 21st until you have received grade from me. After you have received a grade, you may treat the project as yours as continue to work on it as you like.

\section*{The Readme File}

The following text must populate your readme file:
\bigskip
\bigskip

This project was completed by \textbf{insert full name here} in partial fulfilment of  ECON-UB.0232, Data Bootcamp, Fall 2017. I certify that the NYU Stern Honor Code applies to this project. In particular, I have:

Clearly acknowledged the work and efforts of others when submitting written work as
our own. The incorporation of the work of others--including but not limited to their ideas,
data, creative expression, and direct quotations (which should be designated with quotation
marks), or paraphrasing thereof-- has been fully and appropriately referenced using notations
both in the text and the bibliography.

And I understand that:

Submitting the same or substantially similar work in multiple courses, either in the
same semester or in a different semester, without the express approval of all instructors is
strictly forbidden.

I acknowledge that a failure to abide by NYU Stern Honor Code will result in a failing grade for the project and course.

Project Description

\textbf{Insert one paragraph description of your project}

\newpage

\section*{Grading}

Projects will be graded on their overall quality.  This includes, but is not restricted to,
these categories:
%
\begin{itemize}
\item Quality of the idea.  Is the question clearly articulated?  Is it interesting?
Does it have general appeal?
\item Quality of the data.  Does the data support the idea?
Is it the best data for this question?
\item Quality of the code. Is the code readable? Could someone else understand what you were doing and why?
\item Degree of difficulty.  Some ideas are harder than others to implement.
As in Olympic diving, you get credit for taking on a challenge.
\item Professional look.  Does your project look professional?  Are the graphs
easy to understand?  Are they clearly labeled?
\end{itemize}


%\input{../LaTeX/footer.tex}

%\begin{itemize} \itemsep=2\bigskipamount
%\item Clarity of the message
%\item ..
%\end{itemize}


%\input{../LaTeX/footer.tex}

\end{document}
